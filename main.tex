\documentclass[letterpaper,11pt]{article}

\usepackage{latexsym}
\usepackage[empty]{fullpage}
\usepackage{titlesec}
\usepackage{marvosym}
\usepackage[usenames,dvipsnames]{color}
\usepackage{verbatim}
\usepackage{enumitem}
\usepackage[hidelinks]{hyperref}
\usepackage{fancyhdr}
\usepackage[english]{babel}
\usepackage{tabularx}
\usepackage{hyphenat}
\usepackage{fontawesome}
\input{glyphtounicode}


%---------- FONT OPTIONS ----------
% sans-serif
% \usepackage[sfdefault]{FiraSans}
% \usepackage[sfdefault]{roboto}
% \usepackage[sfdefault]{noto-sans}
% \usepackage[default]{sourcesanspro}

% serif
% \usepackage{CormorantGaramond}
% \usepackage{charter}


\pagestyle{fancy}
\fancyhf{} % clear all header and footer fields
\fancyfoot{}
\renewcommand{\headrulewidth}{0pt}
\renewcommand{\footrulewidth}{0pt}

% Adjust margins
\addtolength{\oddsidemargin}{-0.5in}
\addtolength{\evensidemargin}{-0.5in}
\addtolength{\textwidth}{1in}
\addtolength{\topmargin}{-.5in}
\addtolength{\textheight}{1.0in}

\urlstyle{same}

\raggedbottom
\raggedright
\setlength{\tabcolsep}{0in}

% Sections formatting
\titleformat{\section}{
  \vspace{-4pt}\scshape\raggedright\large
}{}{0em}{}[\color{black}\titlerule \vspace{-5pt}]

% Ensure that generate pdf is machine readable/ATS parsable
\pdfgentounicode=1

%-------------------------
% Custom commands

\newcommand{\resumeItem}[1]{
  \item\small{
    {#1 \vspace{-2pt}}
  }
}


\newcommand{\resumeSubheading}[4]{
  \vspace{-2pt}\item
    \begin{tabular*}{0.97\textwidth}[t]{l@{\extracolsep{\fill}}r}
      \textbf{#1} & #2 \\
      \textit{\small#3} & \textit{\small #4} \\
    \end{tabular*}\vspace{-7pt}
}


\newcommand{\resumeSubSubheading}[2]{
    \vspace{-2pt}\item
    \begin{tabular*}{0.97\textwidth}{l@{\extracolsep{\fill}}r}
      \textit{\small#1} & \textit{\small #2} \\
    \end{tabular*}\vspace{-7pt}
}


\newcommand{\resumeEducationHeading}[6]{
  \vspace{-2pt}\item
    \begin{tabular*}{0.97\textwidth}[t]{l@{\extracolsep{\fill}}r}
      \textbf{#1} & #2 \\
      \textit{\small#3} & \textit{\small #4} \\
      \textit{\small#5} & \textit{\small #6} \\
    \end{tabular*}\vspace{-5pt}
}


\newcommand{\resumeProjectHeading}[2]{
    \vspace{-2pt}\item
    \begin{tabular*}{0.97\textwidth}{l@{\extracolsep{\fill}}r}
      \small#1 & #2 \\
    \end{tabular*}\vspace{-7pt}
}

\newcommand{\resumeOrganizationHeading}[4]{
  \vspace{-2pt}\item
    \begin{tabular*}{0.97\textwidth}[t]{l@{\extracolsep{\fill}}r}
      \textbf{#1} & \textit{\small #2} \\
      \textit{\small#3}
    \end{tabular*}\vspace{-7pt}
}

\newcommand{\resumeSubItem}[1]{\resumeItem{#1}\vspace{-4pt}}

\renewcommand\labelitemii{$\vcenter{\hbox{\tiny$\bullet$}}$}

\newcommand{\resumeSubHeadingListStart}{\begin{itemize}[leftmargin=0.15in, label={}]}
\newcommand{\resumeSubHeadingListEnd}{\end{itemize}}
\newcommand{\resumeItemListStart}{\begin{itemize}}
\newcommand{\resumeItemListEnd}{\end{itemize}\vspace{-5pt}}

%-------------------------------------------
%%%%%%  RESUME STARTS HERE  %%%%%%%%%%%%%%%%%%%%%%%%%%%%


\begin{document}

%---------- HEADING ----------

\begin{center}
	\textbf{\Huge \scshape Animesh Nighojkar} \\ \vspace{3pt}
	\small
	\faMobile \hspace{.5pt} \href{tel:16463715208}{+1 646 371 5208}
	$|$
	\faAt \hspace{.5pt} \href{mailto:anighojkar@usf.edu}{anighojkar@usf.edu}
	$|$
	\faLinkedinSquare \hspace{.5pt} \href{https://www.linkedin.com/in/anighojkar}{anighojkar}
	$|$
	\faGithub \hspace{.5pt} \href{https://github.com/coderpotter}{coderpotter}
	$|$
	\faGlobe \hspace{.5pt} \href{https://anighojkar.owlstown.net/}{Portfolio}
\end{center}


%----------- EDUCATION -----------

\section{Education}
\vspace{3pt}
\resumeSubHeadingListStart

\resumeEducationHeading
{University of South Florida}{Tampa, FL, USA}
{Ph.D., Computer Science and Engineering}{Aug 2019 \textbf{--} May 2024}
{Master of Science, Computer Science and Engineering}{Aug 2019 \textbf{--} Dec 2021}

\resumeSubheading
{Rajiv Gandhi Technical University}{Indore, India}
{Bachelor of Engineering, Computer Science and Engineering}{Aug 2015 \textbf{--} May 2019}

\resumeSubHeadingListEnd


%----------- EXPERIENCE -----------

\section{Experience}
\vspace{3pt}
\resumeSubHeadingListStart

\resumeSubheading
{Dropbox, Inc.}{Remote, USA}
{Machine Learning Engineering Intern}{May 2023 \textbf{--} Aug 2023}
\resumeItemListStart
\resumeItem{Led the end-to-end development and deployment of an innovative object removal feature for video editing, utilizing advanced video inpainting and segmentation techniques. This development enabled users to intuitively select and remove objects from videos, with the system automatically tracking and erasing the object across subsequent frames. Collaboration with the front-end team ensured seamless integration into the user interface, significantly enhancing user experience and elevating video editing capabilities.}
\resumeItemListEnd

\resumeSubheading
{Dropbox, Inc.}{Remote, USA}
{Machine Learning Engineering Intern}{May 2022 \textbf{--} Aug 2022}
\resumeItemListStart
\resumeItem{Created and implementated of a novel method to match publicly available annotated data with an unannotated private dataset. Developed and deployed abstractive summarization models for document and video content, enhancing the platform's data utilization efficiency and summarization capabilities in a production environment.}
\resumeItemListEnd

\resumeSubheading
{University of South Florida, AMHR Lab}{Tampa, FL, USA}
{Graduate Research Assistant}{Sep 2019 \textbf{--} present}
\resumeItemListStart
\resumeItem{Prompt Engineering: Designed and executed prompt engineering strategies for GPT-4 and other large language models, creating a specialized dataset with distinct psychological and psychometric properties. This involved meticulous planning and testing to ensure the dataset's relevance and applicability in psychological research.}
\resumeItem{Modeling Human Semantic Fluency with Transformers: Utilized transformer models to simulate human semantic fluency tasks, providing new insights into the intricacies of human cognition and memory retrieval. This work bridged the gap between advanced AI models and cognitive science, offering a novel perspective on how AI can mimic human cognitive processes.}
\resumeItem{Exploring Pre-Trained Language Models like BERT: Conducted in-depth research on pre-trained language models, particularly BERT, to understand their learning mechanisms. This research was pivotal in drawing parallels between machine learning and human learning, contributing to both educational psychology and AI.}
\resumeItem{Advancing Semantic Textual Similarity Research: Identified and addressed a critical gap in existing methods like BERTScore and BLEURT for assessing sentence equivalence. Developed a new method that more accurately captures nuances in Semantic Textual Similarity, enhancing the efficacy of Neural Machine Translation.}
\resumeItemListEnd

\resumeSubheading
{Indian Institute of Technology}{Patna, India}
{Research Intern and Collaborator}{June 2018 \textbf{--} July 2019}
\resumeItemListStart
\resumeItem{Collaborated with the Computer Science Department on developing algorithms for Infrastructure as a Service (IAAS) research projects.}
\resumeItem{Developed an innovative algorithm for energy-efficient VM (Virtual Machine) allocation in data centers, significantly improving upon the existing cubic time complexity algorithms by reducing them to quadratic.}
\resumeItem{Conducted thorough analytic tests to validate the algorithm's efficacy and efficiency in a simulated environment.}
\resumeItemListEnd

\resumeSubHeadingListEnd


%----------- SKILLS -----------

\section{Skills}
\vspace{2pt}
\resumeSubHeadingListStart
\small{\item{
	            \textbf{Languages: }{Python, C++, Java, Rust, JavaScript, SQL, Scala, R, Bash/Shell} \\ \vspace{3pt}
	            \textbf{Technologies: }{Pytorch, TensorFlow, Keras, Pandas, NumPy, Scikit-learn, Matplotlib, Seaborn, Plotly, Git, GitHub, Linux, OpenCV, Django, Flask, Docker, AWS, Google Cloud, Kubernetes, \LaTeX} \\ \vspace{3pt}
	            \textbf{NLP: }{HuggingFace, SpaCy, NLTK, Gensim, StanfordNLP, FastText, BERT, Prompt Enginnering, GPT} \\ \vspace{3pt}
	      }}
\resumeSubHeadingListEnd


%----------- PROJECTS -----------

\section{Projects}
\vspace{3pt}
\resumeSubHeadingListStart

\resumeProjectHeading
{\textbf{Legal Entailment Analysis using Large Language Models}}{}
\resumeItemListStart
\resumeItem{Pioneered prompt engineering techniques using large pre-trained language models (GPT-4 and FLaN-T5), specifically tailored for the legal entailment task (Task 4 in the 2023 COLIEE competition), without relying on fine-tuning these models. Extracted relevant articles and queries from the training set. Utilized an ensemble approach to experiment with various combinations of models and prompts, leading to a more robust and effective solution with an accuracy of $83.3\%$.}
\resumeItemListEnd

\resumeProjectHeading
{\textbf{Research Paper Summarizer}}{}
\resumeItemListStart
\resumeItem{A tool leveraging the capabilities of GPT-4 for efficient and accurate summarization of academic papers. This project involved designing a sophisticated system that could understand and interpret complex scientific text, and then generate concise, pointwise summaries. Special emphasis was placed on ensuring the summaries maintained the core essence and technical accuracy of the original research, while being accessible to a broader audience.}
\resumeItemListEnd

\resumeProjectHeading
{\textbf{Discourse Act Classification and Sarcasm Detection}}{}
\resumeItemListStart
\resumeItem{A two-part text analysis system focusing on advanced natural language processing techniques. The first component was a model designed to classify discourse acts in Reddit threads, identifying one of ten categories such as questions, answers, announcements, agreements, etc. This was achieved without the use of neural networks, relying instead on traditional machine learning algorithms like xgboost, to get an F-1 score of $0.82$. The second component was developing a neural network with Gated Recurrent Unit (GRU) layers to detect sarcasm in conversational text on Reddit, with an F-1 score of $0.89$.}
\resumeItemListEnd

\resumeSubHeadingListEnd


%----------- PUBLICATIONS -----------

\section{Selected Publications}
\vspace{3pt}
\resumeSubHeadingListStart

\resumeSubheading
{\href{https://aclanthology.org/2023.law-1.20/}{\small{No Strong Feelings One Way or Another: Re-operationalizing Neutrality in Natural Language Inference}}}{}
{Proceedings of the 17th Linguistic Annotation Workshop (LAW-XVII)}{}
\resumeItemListStart
\resumeItem{Investigated the operationalization of the neutral label in Natural Language Inference (NLI) datasets, identifying inconsistencies and validity issues that impact language models' ability to mimic human reasoning. Proposed a refined evaluation framework for NLI, uncovering flaws in annotator agreement approaches and highlighting the need for improved methods in the NLP community.}
\resumeItemListEnd

\resumeSubheading
{\href{https://arxiv.org/abs/2208.09719}{\small{Cognitive Modeling of Semantic Fluency Using Transformers}}}{}
{Cognitive Aspects of Knowledge Representation workshop at IJCAI-ECAI}{}
\resumeItemListStart
\resumeItem{Developed an innovative approach using transformer-based language models (TLMs) and hyperparameter hypothesization to model and predict human performance in cognitive tasks, specifically focusing on the Semantic Fluency Task (SFT) in cognitive science. Provided evidence that TLMs can surpass existing computational models in identifying individual differences in cognitive behavior, offering new perspectives on human memory retrieval strategies and advancing the understanding of knowledge representation in cognitive modeling.}
\resumeItemListEnd

\resumeSubheading
{\href{https://aclanthology.org/2021.acl-long.552/}{\small{Improving Paraphrase Detection with the Adversarial Paraphrasing Task}}}{}
{\begin{tabular}[t]{@{}l@{}}Proceedings of the 59th Annual Meeting of the Association for Computational Linguistics and the 11th International \\ Joint Conference on Natural Language Processing (Volume 1: Long Papers)\end{tabular}}{}
\resumeItemListStart
\resumeItem{Introduced the Adversarial Paraphrasing Task (APT), an innovative method for dataset creation in paraphrase identification, emphasizing the generation of semantically equivalent yet lexically and syntactically distinct sentence pairs to enhance model accuracy in detecting sentence-level meaning equivalence. Utilized T5 for automating APT, significantly improving the efficiency of dataset generation and demonstrating the enhanced capability of paraphrase detection models in identifying true semantic equivalences beyond mere lexical and syntactic similarities.}
\resumeItemListEnd

\resumeSubheading
{\href{https://journals.flvc.org/FLAIRS/article/view/128519}{\small{Mutual Implication as a Measure of Textual Equivalence}}}{}
{The International FLAIRS Conference Proceedings, 34}{}
\resumeItemListStart
\resumeItem{Conducted research in Semantic Textual Similarity (STS) and paraphrase detection, focusing on mutual implication (MI) as a novel approach to evaluate sentences based on their inferential properties rather than traditional text comparison methods. Proposed the inclusion of MI as a complementary evaluation metric in diverse NLP fields, such as machine translation and natural language inference, while critically analyzing its limitations and exploring potential methods to address these challenges.}
\resumeItemListEnd

\resumeSubHeadingListEnd


\end{document}
